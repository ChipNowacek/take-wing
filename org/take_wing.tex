% Created 2014-09-23 Tue 11:52
\documentclass[11pt]{article}
\usepackage[utf8]{inputenc}
\usepackage[T1]{fontenc}
\usepackage{fixltx2e}
\usepackage{graphicx}
\usepackage{longtable}
\usepackage{float}
\usepackage{wrapfig}
\usepackage{rotating}
\usepackage[normalem]{ulem}
\usepackage{amsmath}
\usepackage{textcomp}
\usepackage{marvosym}
\usepackage{wasysym}
\usepackage{amssymb}
\usepackage{hyperref}
\tolerance=1000
\usepackage{tawny}
\lstnewenvironment{tawny}{\lstset{style=tawnystyle}}{}
\author{Phillip Lord}
\date{\today}
\title{Take Wing: Building Ontologies with Tawny-OWL}
\hypersetup{
  pdfkeywords={},
  pdfsubject={},
  pdfcreator={Emacs 24.3.1 (Org mode 8.2.7c)}}
\begin{document}

\maketitle
\tableofcontents


\section{Introduction}
\label{sec-1}

This book introduces ontology building using the OWL2 ontology language, and
the Tawny-OWL library. Ontologies are a method for representing knowledge,
generally, but not necessarily, about the world around us. It is then possible
to check that the representation is consistent, as well as drawing conclusions
about new knowledge. They are generally used in complex, knowledge-rich areas
of knowledge, including biomedicine.

Many ontology development tools provide a Graphical User Interface, through
which the ontology developer adds the various entities involved in building an
ontology. However, many ontologies contain large and repetitive sections; for
these, ontology development teams often fall back to generating parts of their
ontology programmatically. Tawny-OWL takes a different approach where ontology
development in a domain-specific language (DSL) embedded in a full programming
language. For structurally simple parts of an ontology, the various components
of an ontology can be specified using the default convienient and simple
Tawny-OWL syntax; for structurally complex parts, new syntax and new patterns
can be built, extending the environment as a core part of ontology
development.

This form of programmatic ontology development is still young. At the moment,
we have used it to produce large ontologies that would have been difficult
using any other technique. However, we also hope that we can also support
easier integration of knowledge-rich structures into applications, so that
ontological data structures can be come a standard part of the programmers
toolkit.


\section{What is an Ontology}
\label{sec-2}
\label{what_is_an_ontology}

Ontologies are about definitions. It is, perhaps, unsurprising therefore
that amount ontologists there are quite a few debates about what exactly
an ontology is and is not; it is not our intention here to either cover
these arguments, nor to give a comprehensive overview of all the uses of
the word.

What is generally agreed is that ontologies describe a set of entities,
in terms of the relationships between these entities, using any of a
number of different relationships. So, for example, we can describe
entities in terms of their class relationships -- what is true of a
superclass is also true of all subclasses. Or we can describe the
\emph{partonomic} relationships: the finger is part of the hand, which is
part of the foot.

An ontology is also very similar to a taxonomy; however, ontologies
place much greater emphasis on their computational properties. This
makes ontologies much more suitable for driving applications and code,
although this often comes at the cost of human understandability of the
ontology. In this document, all the ontologies we talk about are
represented using specific language, called OWL (the Ontology Web
Language). This has very well-defined computational properties, and
through the document we will explore the implications of these
properties.

We also use the term "ontology" to mean a specific object that you can
manipulate in Tawny-OWL -- similar to the way we say that you are
reading some words now.


\section{Environment}
\label{sec-3}

Our hope is that for structurally simple ontologies, Tawny-OWL should be
usable by non-programmers, with a simple and straight-forward syntax.
One area where this hope is currently not fulfilled is right at the
start -- getting a working environment is not as simple as starting an
application such as Protege and programming. In this section, I
introduce the core technology and the basic environment that is needed
to make effective use of Tawny-OWL.


\subsection{The OWL API}
\label{sec-3-1}

Tawny-OWL is built using the \url{http://owlapi.sourceforge.net/[OWL} API].
This library is a comprehensive tool for generating, transforming and
using OWL Ontologies. It is widely used, and is the basis for the
Protege 4 editor. Being based on this library, Tawny-OWL is reliable and
standard-compliant (or at least as reliable and standard-compliant as
Protege!). It is also easy to integrate directly with other tools
written using the OWL API.

\subsection{Clojure}
\label{sec-3-2}

Tawny-OWL is a programmatic library build on top of the Clojure
language. Tawny-OWL takes many things from Clojure. These include:

\begin{itemize}
\item the basic syntax with parentheses and with \texttt{:keywords}
\item the ability to effectively add new syntax
\item the ability to extend Tawny-OWL with patterns
\item integration with other data sources
\item the test environment
\item the build, dependency and deployment tools
\end{itemize}

In addition, most of the tools and environment that Tawny-OWL use to
enable development were built for Clojure and are used directly with
little or no additions. These include:

\begin{itemize}
\item IDEs or editors used for writing Clojure
\item the leiningen build tool
\end{itemize}

Tawny-OWL inherits a line-orientated syntax which means that it works
well with tools written for any programming language; most notable
amoung these are version control systems which enable highly
collaborative working on ontologies.

Clojure is treated as a programmatic library -- the user never starts or
runs Clojure, and there is no \texttt{clojure} command. Rather confusingly,
this role is fulilled by Leiningen, which is the next item on the list.

\subsection{Leiningen}
\label{sec-3-3}

\url{http://www.leiningen.org[Leiningen}] is a tool for working with Clojure
projects. Given a directory structure, and some source code leiningen
will perform many project tasks including checking, testing, releasing
and deploying the project. In addition to these, it has two critical
functions that every Tawny-OWL project will use: first, it manages
dependencies, which means it will download both Tawny-OWL and Clojure;
second, it starts a REPL which is the principle means by which the user
will directly or indirectly interact with Tawny-OWL.

\subsection{REPL}
\label{sec-3-4}

Clojure provides a REPL -- Read-Eval-Print-Loop. This is the same things
as a shell, or command line. For instance, we can the following into a
Clojure REPL, and it will print the return value, or 2 in this case.


\begin{tawny}
;; returns 2 
(+ 1 1)
\end{tawny}

The most usual way to start a REPL is to use leiningen, which then sets
up the appropriate libraries for the local project. For example,
\texttt{lein repl} in the source code for this document, loads a REPL with
Tawny-OWL pre-loaded.

In practice, most people use the REPL indirectly through their IDE.

\subsection{IDE or Editor}
\label{sec-3-5}

Clojure is supported by a wide variety of editors, which in turn means
that they can be used for Tawny-OWL. The choice of an editor is a very
personal one (I use Emacs), but in practice any good editor will work.

The editor has two main roles. Firstly, as the name suggests it provides
a rich environment for writing Tawny-OWL commands. Secondly, the IDE
will start and interact with a REPL for you. This allows you to add or
remove new classes and other entities to an ontology interactively.
Tawny-OWL has been designed to take advantage of an IDE environment; in
most cases, for example, auto-completion will happen for you.

\subsection{Further Information}
\label{sec-3-6}

There are many sources of further information about Clojure which will
be listed here.


\section{Getting Started}
\label{sec-4}

In this section, we will build the most ontology and start to show the
basic capabilities of Tawny-OWL.

As described in \label{/the/environment-the-environment}, Tawny-OWL can be
used with several different toolchains. In this section, we will run
through the building a very simple ontology. There is an <> describing
how to achieve each of these steps with specific tool chains.

\subsection{Getting a Project}
\label{sec-4-1}

For this book, we will use a pre-rolled project -- in fact the one used
to create this book. You can access the project data from
\url{https://github.com/phillord/take-wing[github}], either using \texttt{git} or
through the download option. If you wish to know how to build a project
yourself, please read <>.

A leiningen project is, essentially, a directory structure with a
project file. The \texttt{project.clj} file for this book looks like this:

\begin{tawny}
(defproject take-wing "0.1.0-SNAPSHOT"
  :dependencies [[uk.org.russet/tawny-owl "1.1.1-SNAPSHOT"]])
\end{tawny}

This includes three critical pieces of information. Firstly \texttt{take-wing}
which is the name of the project. Secondly, immediately after this is a
version number such as \texttt{0.1.0}. Finally, we have a \texttt{:dependencies} which
includes only a single dependency to \texttt{tawny-owl} itself.


\subsection{Starting a new ontology}
\label{sec-4-2}

It is possible to build an ontology in Tawny-OWL using almost no
functions from Clojure, the language on which it is built; the only
necessary exception to this is the \emph{namespace declaration}.

Like most programming languages, Clojure has a namespacing mechanism.
These are declared at the start of each file, and the namespace relates
to the file name and location. Finally, the namespace form is also used
to import functions from other namespaces. Here, we define a namespace
called \texttt{take.wing.getting-started} which would be defined in a file
\texttt{take/wing/getting\_started.clj}. Secondly, we import \texttt{tawny.owl}
namespace which contains the core functions of Tawny-OWL.

\begin{tawny}
(ns take.wing.getting-started (:use [tawny.owl]))
\end{tawny}

Tawny-OWL also uses the namespace mechanism to define the scope of an
ontology. In general, an ontology is defined within a single namespace,
and each namespace defines a single ontology. A new ontology is declared
with the \texttt{defontology} form. This also introduces a new symbol, \texttt{pizza},
which can be used latter to refer.

\begin{tawny}
(defontology pizza)
\end{tawny}


\subsection{Connecting to a repl}
\label{sec-4-3}

Currently, the source for the ontology has been created, but this is not
"live" -- for this, we must start a Clojure process and connect to it
via a REPL and then evaluate the file. With the current contents, the
REPL should show something like the following which is the result of
evaluating the last form, and shows that we have defined a new symbol
footnote:[In Clojure, it is actually a var that has been created]

\begin{verbatim}
=> #'take.wing.getting-started/pizza
\end{verbatim}

The symbol \texttt{pizza} now refers to an object live in the system. If
evalulate \texttt{pizza}, a hopefully informative string message will be
printed.


\begin{verbatim}
take.wing.getting-started> pizza
#<OWLOntologyImpl Ontology(OntologyID(OntologyIRI(<#pizza>))) [Axioms: 0 Logical Axioms: 0]>
\end{verbatim}

\subsection{Creating some entities}
\label{sec-4-4}

Now we create some entities for our ontology, in this case two classes
called \texttt{Pizza} and \texttt{MargheritaPizza}, and state that \texttt{MargheritaPizza}
is a subclass of \texttt{Pizza}. This forms implicitly place these two terms

\begin{tawny}
(defclass Pizza)
(defclass MargheritaPizza :super Pizza)
\end{tawny}

Finally, we are in a position to make a useful query against this which
we can do using the \texttt{subclasses} function.

\begin{tawny}
(subclasses Pizza)
\end{tawny}

In a REPL session, this returns a set with one element ---
\texttt{MargheritaPizza}. If we evaluate \texttt{pizza} (the ontology) again, we also
see that the ontology now has a number of axioms.


\begin{tawny}
(subclasses Pizza)
#{#<OWLClassImpl <#pizza#MargheritaPizza>>}
take.wing.getting-started> pizza
=> #<OWLOntologyImpl Ontology(OntologyID(OntologyIRI(<#pizza>))) [Axioms: 5 Logical Axioms: 1]>
\end{tawny}


\subsection{Summary}
\label{sec-4-5}

In this section, we have outlined the basic tasks that are needed to
build ontologies with Tawny-OWL: creating a project, creating an
ontology, creating some entities. We have also started to show how to
use and query over them. In the next section, we will build this
ontology in full, using it to demonstrate many parts of Tawny-OWL and
OWL ontologies in general.





\section{The Pizza Ontology}
\label{sec-5}

In this section, we will create a Pizza ontology; we choose pizzas because
they are simple, well-understood and compositional (see \href{http://robertdavidstevens.wordpress.com/2010/01/22/why-the-pizza-ontology-tutorial/}{here} for more).

We have described ontologies more abstractly \hyperref[what_is_an_ontology]{earlier}. More concretely, in this
book, an ontology is a computational object, which can contain a number of
different objects. These objects can be of several different kinds. The most
(and least!) important of these are \emph{individuals}. We say that these are the
most important because it is these individuals that are described and
constrained by the other objects. We say that they are the least important
because, in practice, many ontologies do not explicitly describe any
individuals at all.

If this seems perverse, consider a menu in a pizza shop. We might seem
an item described saying "Margherita\ldots{}.£5.50". The menu makes no
statements at all about an individual pizza. It is saying that any
margherita pizza produced in this resturant is going to (or already has)
cost £5.50. From the menu, we have no idea how many margherita pizzas
have been produced or have been consumed. But, menu is still useful. The
menu is comprehensive, tells you something about all the pizzas that
exist (at least in one resturant) and the different types of pizza. This
is different to the bill, which describes individuals -- the pizzas that
have actually been provided, how many pizza and how much they all cost.
In ontology terms, the menu describes the \textbf{classes}, the bill describes
individuals \footnote{The analogy between a pizza menu and an ontology
is not perfect. With pizza, people are generally happy with the classes
(i.e. the menu) and start arguing once about the individuals (i.e. the
bill); with ontologies it tends to be the other way around}. OWL
Ontologies built with Tawny-OWL \emph{can} describe either or both of these
entities but in most cases focus on classes.


\subsection{Defining Classes}
\label{sec-5-1}

We start with a namespace form, and a \verb~use~ statement for \verb~tawny.owl~,
and a statement declaring a new ontology. First, consider the syntax of
this example, because it is shared by all statements in Tawny-OWL. All
expresions in Clojure are delimited by \verb~(~ and \verb~)~ and Tawny-OWL follows
this rule. Next, we have a name for the object we wish to create -- in
this case an new ontology. This starts with \verb~def~ to indicate that we
also want to create a new symbol which we can use to refer to this
entity later.

Finally, come a set of arguments, introduced with \emph{keywords}. These all
end with a \verb~:~. In this case, \verb~:iri~ introduces the main IRI for this
ontology, which is a global identifier, and finally a string which is
the actual value of that argument.

\begin{tawny}
(ns take.wing.the-pizza-ontology (:use [tawny.owl]))

(defontology pizza :iri "http://purl.org/ontolink/take-wing/pizza")
\end{tawny}


The semantics of this statement are quite interesting. If we had created
a new database, by default, the database would be considered to be empty
-- that is there would be no individuals in it. With an ontology, the
opposite is true. By default, we assume that there could be any number
of individuals. As of yet, we just have not said anything about these
individuals.

Next, we declare two classes. A class is a set of individuals with
shared characteristics. For now, we create two classes, \verb~Pizza~ and
\verb~PizzaComponent~. As with our \verb~defontology~ form, have a \verb~def~ form;
however, in this case, we do not use any arguments. The semantics of
these two statements are that, there is a class called \verb~Pizza~ and
another called \verb~PizzaComponent~ which individuals may be members of.
However, we know nothing at all about the relationship between an
individual \verb~Pizza~ and an individual \verb~PizzaComponent~.


\begin{tawny}
(defclass Pizza) 
(defclass PizzaComponent)
\end{tawny}

To build an accurate ontology, we may wish to describe this relationship
further. We might ask the question, can an individual be both a \verb~Pizza~
and a \verb~PizzaComponent~ at the same time. The answer to this is no, but
currently our ontology does not state this. In OWL terminology, we wish
to say that these two classes are \emph{disjoint}. We can achieve this by
adding an \verb~as-disjoint~ statement.

\begin{tawny}
(as-disjoint Pizza PizzaComponent)
\end{tawny}

This works well, but is a little duplicative. If we add a new class
which we wish to also be disjoint, it must be added in two places.
Instead, it is possible to do both at once \footnote{In the source code,
generated from this book, we are now defining both classes twice, as we
have two \verb~defclass~ statements for each. This will actually work okay,
although it is not best practice as it is somewhat dependent on the
implementation details of the OWL API.}. This has the advantage of
grouping the two classes together in the file, as well as semantically,
which should make the source more future-proof; should we need new
classes, we will automatically make them disjoint as required.

\begin{tawny}
(as-disjoint
 (defclass Pizza)
 (defclass PizzaComponent))
\end{tawny}

The semantics of these statements are that our ontology may have any
number of individuals, some of which may be \verb~Pizza~, some of which may
be \verb~PizzaComponent~, but none of which can be both \verb~Pizza~ and
\verb~PizzaComponent~ at the same time. Before we added the \verb~as-disjoints~
statement, we would have assumed that it was possible to be both.

As well as describing that two classes are different, we may also wish
to describe that they are closely related, or that they are
\emph{subclasses}. Where one class is a subclass of another, we are saying
that everything that is true of the superclass is also true of the
subclass. Or, in terms of individuals, that every individual of the
subclass is also an individual of the superclass.

Next, we add two more classes and include the statement that they have
\verb~PizzaComponent~ as a superclass. We do this by adding a \verb~:super~
argument or \emph{frame} to our \verb~defclass~ statement. In Tawny-OWL the frames
can all be read in the same way. Read forwards, we can say \verb~PizzaBase~
has a superclass \verb~PizzaComponent~, or backwards \verb~PizzaComponent~ is a
superclass of \verb~PizzaBase~. Earlier, we say the \verb~:iri~ frame for
\verb~defontology~ which is read similarly -- \verb~pizza~ has the given IRI.

As every individual of, for example, \verb~PizzaBase~ is \verb~PizzaComponent~, and no
\verb~PizzaComponent~ individual can also be a \verb~Pizza~ this also implies that no
\verb~PizzaBase~ is a \verb~Pizza~. In otherwords, the disjointness is inherited
\footnote{In this ontology, we use a naming scheme using CamelCase, upper case
names for classes and, later, lower case properties. As with many parts of
ontology development, opinions differ as to whether this is good. With
Tawny-OWL it has the fortuitous advantage that it syntax-highlights nicely,
because it looks like Java}

\begin{tawny}
(defclass PizzaBase
  :super PizzaComponent)
(defclass PizzaTopping
  :super PizzaComponent)
\end{tawny}


As with the disjoint statement, this is little long winded; we have to name
the \verb~PizzaComponent~ superclass twice. Tawny-OWL provides a short cut for
this, with the \verb~as-subclasses~ function.

\begin{tawny}
(as-subclasses
 PizzaComponent
 (defclass PizzaBase)
 (defclass PizzaTopping))
\end{tawny}

We are still not complete; we asked the question previously, can you be both a
\verb~Pizza~ and a \verb~PizzaComponent~, to which the answer is no. We can apply the
same question, and get the same answer to a \verb~PizzaBase~ and \verb~PizzaTopping~.
These two, therefore, should also be disjoint. However, we can make a stronger
statement still. The only kind of \verb~PizzaComponent~ that there are either a
\verb~PizzaBase~ or a \verb~PizzaTopping~. We say that the \verb~PizzaComponent~ class is
\emph{covered} by its two subclasses. We can add both of these statements to the
ontology also.

\begin{tawny}
(as-subclasses
 PizzaComponent
 :disjoint :cover
 (defclass PizzaBase)
 (defclass PizzaTopping))
\end{tawny}

We now have the basic classes that we need to describe a pizza.

\subsection{Properties}
\label{sec-5-2}

Now, we wish to describe more about \verb~Pizza~; in particular, we want to say
more about the relationship between \verb~Pizza~ and two \verb~PizzaComponent~ classes.
OWL provides a rich mechanism for describing relationships between individuals
and, in turn, how individuals of classes are related to each other. As well as
there being many different types of individuals, there are can be many
different types of relationships. It is the relationships to other classes or
individuals that allow us to describe classes, and it is for this reason that
the different types of relationships are called \emph{properties}.

A \verb~Pizza~ is built from one or more \verb~PizzaComponent~ individuals; we first
define two properties \footnote{Actually, two \emph{object} properties, hence
\verb~defoproperty~. We can also define \emph{data} properties, which we will see later}
to relate these two together, which we call \verb~hasComponent~ and
\verb~isComponentOf~. The semantics of this statement is to say that we now have
two properties that we can use between individuals.

\begin{tawny}
(defoproperty hasComponent)
(defoproperty isComponentOf)
\end{tawny}

As with classes, there is more that we can say about these properties. In this
case, the properties are natual opposites or inverses of each other. The
semantics of this statement is that for an individual \verb~i~ which \verb~hasComponent~
\verb~j~, we can say that \verb~j~ \verb~isComponentOf~ \verb~i~ also. 

\begin{tawny}
(as-inverse
 (defoproperty hasComponent)
 (defoproperty isComponentOf))
\end{tawny}

Again, the semantics here are actually between individuals, rather than
classes. This has an important consequence with the inverses. We might make
the statement that \verb~Pizza~ \verb~hasComponent~ \verb~PizzaComponent~, but this does not
allow us to infer that \verb~PizzaComponent~ \verb~isComponentOf~ \verb~Pizza~. Using an
every day analogy, just because all bicycles have wheels, we can not assume
that all wheels are parts of a bike; we \textbf{can} assume that where a bike has a
wheel, that wheel is part of a bike. This form of semantics is quite subtle,
and is an example of where statements made in OWL are saying less than most
people would assume footnote:[We will see examples of the opposite also --
statements which are stronger in OWL than the intuitive interpretation].

We now move on to describe the relationships between \verb~Pizza~ and both of
\verb~PizzaBase~ and \verb~PizzaTopping~. For this, we will introduce three new parts of
OWL: subproperties, domain and range constraints and property characteristics,
which we define in Tawny-OWL as follows:


\begin{tawny}
(defoproperty hasTopping
  :super hasComponent
  :range PizzaTopping
  :domain Pizza)

(defoproperty hasBase
  :super hasComponent
  :characteristic :functional
  :range PizzaBase
  :domain Pizza)
\end{tawny}


First, we consider sub-properties, which are fairly analogous to sub-classes.
For example, if two individuals \verb~i~ and \verb~j~ are related so that \verb~i hasTopping j~, then it is also true that \verb~i hasComponent j~.

Domain and range constraints describe the kind of entity that be at either end
of the property. So, for example, considering \verb~hasTopping~, we say that the
domain is \verb~Pizza~, so only instances of \verb~Pizza~ can have a topping, while the
range is \verb~PizzaTopping~ so only instances of \verb~PizzaTopping~ can be a topping. 

Finally, we introduce a \emph{characteristic}. OWL has quite a few different
characteristics which will introduce over time; in this case \emph{functional}
means means that there can be only one of these, so an individual has only a
single base.


\subsection{Populating the Ontology}
\label{sec-5-3}

We now have enough expressivity to describe quite a lot about pizzas. So, we
can now set about creating a larger set of 

\begin{tawny}
(as-subclasses
 :disjoint
 PizzaTopping
 (defclass CheeseTopping)
 (defclass FishTopping)
 (defclass FruitTopping)
 (defclass HerbSpiceTopping)
 (defclass MeatTopping)
 (defclass NutTopping)
 (defclass SauceTopping)
 (defclass VegetableTopping))
\end{tawny}




\section{Appendix}
\label{sec-6}

\subsection{Appendix: Tool Specific Instructions}
\label{sec-6-1}

In this appendix with give specific instructions for achieving the tasks
in the \label{/getting/started-Getting-Started} section and elsewhere. These
instructions are relatively brief pointers. In practice, you may need to
consult the documentation of the various systems. Each section begins
with a short comment on the advantages and disadvantages of each system.

\begin{itemize}
\item <>
\item <>
\item <>
\item <>
\end{itemize}

\subsubsection{Emacs}
\label{sec-6-1-1}

Emacs provides a rich environment for editing Tawny-OWL ontologies,
especially when combined with a set of other tools used from the command
line. Emacs is itself written in Lisp. It is fantastic for
keyboard-lovers as everything can be done from the keyboard. Emacs is a
tool with a long history which means that it tends to conform to its own
standards rather than those of the operating systems it works on, which
can make it strange experience for those new to it.

I (Phillip Lord) use Emacs and a few of the tools described in this
document are Emacs-specific.

\begin{enumerate}
\item Installation
\label{sec-6-1-1-1}

Emacs can be installed with a system-specific package-manager. You will
need at least Emacs 24. Once installed you will need to add two Emacs
packages to get a rich Clojure environment: these are
\url{https://github.com/clojure-emacs/cider[=cider}=] and
\url{https://github.com/clojure-emacs/clojure-mode[=clojure-mode}=]. In
addition, there are a set of supporting systems which add to the
experience of using Emacs for Tawny-OWL: these include
\url{http://github.com/magit/magit[=magit}=] for controlling
\url{http://git-scm.org[git}];
\url{https://github.com/bbatsov/projectile[=projectile}=] for navigating
projects quickly; and \url{http://www.emacswiki.org/emacs/ParEdit[=paredit}=]
which keeps parentheses in place.

In addition to Emacs, \url{http://leiningen.org[leiningen}] needs to be
installed for project management. Finally, \url{http://git-scm.org[git}] is
used for versioning.

\item Getting a project
\label{sec-6-1-1-2}

To get the take-wing project, just clone the git repository with git
clone. Alternatively, without git download the zip file and unpack.
Emacs does not require that the project be imported; opening a file
inside a project is all that is needed.

\begin{verbatim}
git clone https://github.com/phillord/take-wing
\end{verbatim}


\item Starting a New Namespace
\label{sec-6-1-1-3}

Emacs does not need specific support for this. Simply open a new file.
If you have installed Clojure mode and Cider correctly, Emacs will
insert a namespace form for you. You will need to add \texttt{:use} statement
by hand.

\item Starting a REPL
\label{sec-6-1-1-4}

Emacs uses Leiningen to start a REPL so this must be installed. It is
launched directly by cider. To do so type \texttt{M-x cider-jack-in}, which
defaults to \texttt{C-c M-j}. It takes a short while to work (especially the
first time, since Leiningen has to download dependencies).

\item Eval a form
\label{sec-6-1-1-5}

Assuming that a REPL has been started, it is possible to evaluate forms
in place; look for the "eval" menu items under the "Cider" menu. As well
as individual forms and a region, whole files can be evaluated. The
results of an the evaluaton appear in the minibuffer. It is also
possible to interact directly with the REPL buffer which is useful for
playing.
\end{enumerate}

\subsubsection{Eclipse}
\label{sec-6-1-2}

Eclipse is a modern IDE. This means that it is full of functionality,
conforms to modern standards and works with lots of languages.
Conversely, it requires lots of resources and forces you to work its
way.

\subsubsection{Vim}
\label{sec-6-1-3}

Vim, like <> is an editor with a long history. It is fast, light-weight
and extremely functional. Also like Emacs, it can be a bit jarring at
first. It has very good Clojure support.

\subsubsection{Lighttable}
\label{sec-6-1-4}

Lighttable is a very new IDE whose purpose is to reinvent many of the
existing idioms associated with programming. The demos are visually
impressive.



\section{Incomplete Import}
\label{sec-7}




include::asciidoc/literate.adoc[]
include::asciidoc/how$_{\text{is}}$$_{\text{this}}$$_{\text{written}}$.adoc[]
include::asciidoc/tool$_{\text{support}}$.adoc[]
include::asciidoc/a$_{\text{new}}$$_{\text{project}}$.adoc[]
% Emacs 24.3.1 (Org mode 8.2.7c)
\end{document}
