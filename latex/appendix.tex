
\section{Appendix}
\label{sec-6}

\subsection{Appendix: Tool Specific Instructions}
\label{sec-6-1}

In this appendix with give specific instructions for achieving the tasks
in the \label{/getting/started-Getting-Started} section and elsewhere. These
instructions are relatively brief pointers. In practice, you may need to
consult the documentation of the various systems. Each section begins
with a short comment on the advantages and disadvantages of each system.

\begin{itemize}
\item <>
\item <>
\item <>
\item <>
\end{itemize}

\subsubsection{Emacs}
\label{sec-6-1-1}

Emacs provides a rich environment for editing Tawny-OWL ontologies,
especially when combined with a set of other tools used from the command
line. Emacs is itself written in Lisp. It is fantastic for
keyboard-lovers as everything can be done from the keyboard. Emacs is a
tool with a long history which means that it tends to conform to its own
standards rather than those of the operating systems it works on, which
can make it strange experience for those new to it.

I (Phillip Lord) use Emacs and a few of the tools described in this
document are Emacs-specific.

\begin{enumerate}
\item Installation
\label{sec-6-1-1-1}

Emacs can be installed with a system-specific package-manager. You will
need at least Emacs 24. Once installed you will need to add two Emacs
packages to get a rich Clojure environment: these are
\url{https://github.com/clojure-emacs/cider[=cider}=] and
\url{https://github.com/clojure-emacs/clojure-mode[=clojure-mode}=]. In
addition, there are a set of supporting systems which add to the
experience of using Emacs for Tawny-OWL: these include
\url{http://github.com/magit/magit[=magit}=] for controlling
\url{http://git-scm.org[git}];
\url{https://github.com/bbatsov/projectile[=projectile}=] for navigating
projects quickly; and \url{http://www.emacswiki.org/emacs/ParEdit[=paredit}=]
which keeps parentheses in place.

In addition to Emacs, \url{http://leiningen.org[leiningen}] needs to be
installed for project management. Finally, \url{http://git-scm.org[git}] is
used for versioning.

\item Getting a project
\label{sec-6-1-1-2}

To get the take-wing project, just clone the git repository with git
clone. Alternatively, without git download the zip file and unpack.
Emacs does not require that the project be imported; opening a file
inside a project is all that is needed.

\begin{verbatim}
git clone https://github.com/phillord/take-wing
\end{verbatim}


\item Starting a New Namespace
\label{sec-6-1-1-3}

Emacs does not need specific support for this. Simply open a new file.
If you have installed Clojure mode and Cider correctly, Emacs will
insert a namespace form for you. You will need to add \texttt{:use} statement
by hand.

\item Starting a REPL
\label{sec-6-1-1-4}

Emacs uses Leiningen to start a REPL so this must be installed. It is
launched directly by cider. To do so type \texttt{M-x cider-jack-in}, which
defaults to \texttt{C-c M-j}. It takes a short while to work (especially the
first time, since Leiningen has to download dependencies).

\item Eval a form
\label{sec-6-1-1-5}

Assuming that a REPL has been started, it is possible to evaluate forms
in place; look for the "eval" menu items under the "Cider" menu. As well
as individual forms and a region, whole files can be evaluated. The
results of an the evaluaton appear in the minibuffer. It is also
possible to interact directly with the REPL buffer which is useful for
playing.
\end{enumerate}

\subsubsection{Eclipse}
\label{sec-6-1-2}

Eclipse is a modern IDE. This means that it is full of functionality,
conforms to modern standards and works with lots of languages.
Conversely, it requires lots of resources and forces you to work its
way.

\subsubsection{Vim}
\label{sec-6-1-3}

Vim, like <> is an editor with a long history. It is fast, light-weight
and extremely functional. Also like Emacs, it can be a bit jarring at
first. It has very good Clojure support.

\subsubsection{Lighttable}
\label{sec-6-1-4}

Lighttable is a very new IDE whose purpose is to reinvent many of the
existing idioms associated with programming. The demos are visually
impressive.


